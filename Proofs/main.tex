\documentclass[12pt]{article}
\usepackage[utf8]{inputenc}

\usepackage{amssymb}
\usepackage{amsmath}
\usepackage{amsthm}
\usepackage{array}
\usepackage{listings}
\usepackage{algpseudocode}

\usepackage{tkz-graph}
\usetikzlibrary{positioning}

\title{CS 271 - Project 5}
\author{Nicholas Reichert, Oscar Martinez}
\date{October 23, 2019}

\begin{document}

\maketitle

\section{}

Prove a complete binary tree with height $h$ contains $2^{h+1} - 1$ total nodes.

\begin{proof}
We will prove this statement using induction.
\bigskip

\textbf{Inductive Hypothesis: } A complete binary tree with height $h$ contains $2^{h+1} - 1$ total nodes.

\bigskip
\textbf{Base Case: } The simplest binary tree would be the binary tree with no nodes.  However, that binary tree has no root, so its height is not defined. (The height of a tree is defined as the height of its root.)

The next simplest binary tree would be the binary tree with just a root.  The tree would have height $0$, since the root has no children.
\begin{equation}
    2^{h+1} - 1 = 2^{0+1} - 1 = 2^1 - 1 = 2 - 1 = 1
\end{equation}

Our hypothesis for $h=0$ is correct, because in total, this tree has $1$ node.  

\bigskip
\textbf{Inductive Step: }

Assume for the sake of induction that a complete binary tree with height $h$ contains $2^{h+1} - 1$ total nodes.

Now, imagine a complete binary tree with height $h+1$.  It would have two children, each of which themselves are the roots of complete binary trees with height $h$.  They each have $2^{h+1} - 1$ total nodes.  The total number of nodes in this tree is
\begin{align}
    (2^{h+1} - 1) + (2^{h+1} - 1) + 1
\end{align}
because the number of nodes in this tree is the sum of the number of nodes in its children trees, plus one node for the root itself.  Simplifying, we get
\begin{align}
    (2^{h+1} - 1) + (2^{h+1} - 1) + 1 &= 2(2^{h+1}) - 1 \\
    &= 2^{h+2} - 1 \\
    &= 2^{(h+1)+1} - 1
\end{align}

A complete binary tree with height $h+1$ has $2^{(h+1)+1} - 1$ total nodes.  So by induction the hypothesis holds for all values of $h$.

\end{proof}
\setcounter{equation}{0}

\section{}

Prove a complete binary tree with $n$ nodes has $(n - 1)/2$ internal nodes.

\begin{proof}
We will prove this statement directly.
\bigskip

Suppose you have a complete binary tree with $n$ total nodes.  Every complete binary tree has a number of nodes of the form $2^{k}-1$, where $k$ is an integer (this follows from \#1 which was proved above).  For this binary tree, 
\begin{align}
    n &= 2^{k}-1 \\
    n+1 &= 2^k \\
    \log_{2}{(n+1)} &= k
\end{align}

We can add $1$ to $k$ to get the number of nodes in the next bigger complete binary tree.
\begin{align}
    2^{k+1} &= 2^{\log_{2}{(n+1)} + 1}-1 \\
    &= 2(2^{\log_{2}{(n+1)}}) - 1 \\
    &= 2(n+1) - 1
\end{align}

This bigger binary tree will have $2(n+1) - 1 = 2n + 1$ total nodes.

This means that we added
\begin{equation}
    2n+1 - n = n+1
\end{equation}

$n+1$ nodes to the new tree as leaves.  So this bigger binary tree has $n+1$ leaves and $(2n+1) - (n+1) = n$ internal nodes.

Let $a$ be the total number of nodes in this new tree.
\begin{equation}
    a = n+1 + n = 2n+1
\end{equation}

It has $n$ internal nodes.  Writing $n$ in terms of $a$, we get.
\begin{align}
    a &= 2n+1 \\
    a-1 &= 2n \\
    (a-1)/2 &= n
\end{align}

So a complete binary tree with $a$  nodes has $(a-1)/2$ internal nodes.

\end{proof}
\setcounter{equation}{0}


\end{document}
